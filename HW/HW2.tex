\documentclass[12pt,letterpaper, onecolumn]{exam}
\usepackage{amsmath}
\usepackage{tikz}
\usepackage{tikz-cd}
\usepackage{amsthm}

% Number equations by question (nice for homework)
\numberwithin{equation}{question}

% Theorem-like environments, numbered per question
\newtheorem{theorem}{Theorem}[question]
\newtheorem{lemma}[theorem]{Lemma}
\theoremstyle{definition}
\newtheorem{definition}[theorem]{Definition}
\theoremstyle{remark}
\newtheorem{remark}{Remark}

% Make the QED symbol a solid square (optional)
\renewcommand{\qedsymbol}{$\blacksquare$}
\newcommand{\Q}[0]{\mathbb{Q}}
\newcommand{\N}[0]{\mathbb{N}}
\newcommand{\Z}[0]{\mathbb{Z}}
\newcommand{\R}[0]{\mathbb{R}}

\usepackage{amssymb}
\usepackage[lmargin=71pt, tmargin=1.2in]{geometry}  %For centering solution box
\lhead{ }
\rhead{ }
% \chead{\hline} % Un-comment to draw line below header
\thispagestyle{empty}   %For removing header/footer from page 1

\begin{document}

\begingroup  
    \centering
    \LARGE Anslysis\\
    \LARGE Problem Set 2\\[0.5em]
    \large \today\\[0.5em]
    \large Xuyi (Sam) Ren\par
    % \large Roll Number\par
    \large MAT-316\par
\endgroup
\rule{\textwidth}{0.4pt}
\pointsdroppedatright   %Self-explanatory
\printanswers
\renewcommand{\solutiontitle}{\noindent\textbf{Ans:}\enspace}   %Replace "Ans:" with starting keyword in solution box
\begin{questions}


\question[4d] Prove sequence $a_n=\frac{n!}{2^n}$ diverges to positive infinity.

\begin{solution}
    We must show that for every $M > 0$, there exists an $N \in \mathbb{N}$ such that $n \ge N$ implies $a_n \ge M$.

    First, we prove the inequality $n! \ge 4^{n-2}$ for all $n \ge 3$.
    \begin{proof}
        We prove by induction on $n$.
        \begin{itemize}
            \item Base case\textbf{ ($n=3$):} We have $3! = 6$ and $4^{3-2} = 4^1 = 4$. Since $6 \ge 4$, the base case holds.
            \item Inductive step: Suppose that $k! \ge 4^{k-2}$ for some $k \ge 3$. We wish to show $(k+1)! \ge 4^{(k+1)-2}$.
            
            Observe that:
            \begin{align*}
                (k+1)! &= (k+1) \cdot k! \\
                &\ge (k+1) \cdot 4^{k-2} \quad (\text{by the inductive hypothesis})
            \end{align*}
            Since $k \ge 3$, we have $k+1 \ge 4$. Therefore:
            \[ (k+1) \cdot 4^{k-2} \ge 4 \cdot 4^{k-2} = 4^{k-1} = 4^{(k+1)-2}. \]
            Thus, the inequality holds for $k+1$.
        \end{itemize}
    \end{proof}

    Using this inequality, we can bound the sequence $a_n$ for $n \ge 3$:
    \begin{equation*}
        a_n = \frac{n!}{2^n} \ge \frac{4^{n-2}}{2^n} = \frac{(2^2)^{n-2}}{2^n} = \frac{2^{2n-4}}{2^n} = 2^{n-4}.
    \end{equation*}
    
    Now, let $M > 0$ be arbitrary. We seek $N$ such that $2^{n-4} \ge M$. Solving for $n$:
    \begin{align*}
        2^{n-4} &\ge M \\
        n - 4 &\ge \log_2 M \\
        n &\ge \log_2 M + 4.
    \end{align*}
    
    Choose $N = \max\{3, \lceil \log_2 M + 4 \rceil\}$.
    Then, for all $n \ge N$, we have $n \ge 3$ (so the inductive bound applies) and $n \ge \log_2 M + 4$.
    This implies:
    \[ a_n \ge 2^{n-4} \ge 2^{(\log_2 M + 4) - 4} = 2^{\log_2 M} = M. \]
    Therefore, $\lim_{n \to \infty} a_n = +\infty$. 
\end{solution}


\question[] 
Before we gave the formal definition of convergence, we said intuitively that a sequence
$\{a_n\}$ converges to $a$ if the $a_n$ get ``closer and closer'' to $a$.
Did we mean that each successive term was closer to $a$ than the previous one?
No. We used ``closer and closer'' in an imprecise way.
This illustrates why one needs technical definitions which say exactly what one means.


\begin{enumerate}
    \item Find a sequence $\{a_n\}$ and a real number $a$ such that
    \[
        |a_{n+1} - a| < |a_n - a|
    \]
    for each $n$, but $\{a_n\}$ does not converge to $a$.
    Thus a sequence can get ``closer and closer'' to $a$ without converging to $a$.

    \item Find a sequence $\{a_n\}$ and a real number $a$ such that $a_n \to a$ but such that
    the above inequality is violated for infinitely many $n$.
    Thus a sequence can converge without getting ``closer and closer.''
\end{enumerate}
\begin{solution}
\textbf{Question 1:} 
Since we need to find a sequence that is getting closer and closer to a value each term but do not converge.

Namely, we can not choose a $N$ s.t. for any $\varepsilon>0$ ,  $n\ge N$  we have $|a_n-a|\le \varepsilon$. That is there must have some term that is outside the $\varepsilon$ ball of $a$. 

So far we have two types of divergence, one is diverge to infinite which would not be our choice since it's not getting close to any finite value. Thus, we have the "oscillation" divergence to work on.

Consider the sequence
\[
    a_n = (-1)^n\left(\frac{1+n}{n}\right)
\]
with $a=0$. Then $\{a_n\}$ does not converge to $0$ (it oscillates), but it does satisfy
\[
    |a_{n+1}-0|<|a_n-0| \quad \text{for every } n.
\]
Indeed,
\[
    |a_n-0| = \left|(-1)^n\left(\frac{1+n}{n}\right)\right| = \frac{1+n}{n},
\]
so
\[
    |a_{n+1}-0| = \frac{2+n}{n+1} < \frac{1+n}{n} = |a_n-0|.
\]
However, $\{a_n\}$ does not converge to $0$ because $|a_n-0|\ge 1$ for all $n$, so taking $\varepsilon=\tfrac12$ we have $|a_n-0|\ge \varepsilon$ for every $n$; hence there is no $N$ such that $n\ge N$ implies $|a_n-0|<\varepsilon$.



\textbf{Quesiton 2:}

Such a sequence that converge to a value without getting closer and closer to its value each term could be constant function from $\N$ to $\R$.\begin{equation*}
    a_n=1^n
\end{equation*}
where this sequence indeed converge to 1. Suppose $|a_n-1|\le \varepsilon$
\begin{align*}
    |a_n-1| &\le \varepsilon \\
    |1^n-1| &\le \varepsilon \\
    0\le \varepsilon
\end{align*}
Thus, $|a_n-1|\le \varepsilon$ holds for all $n$, therefore we shall choose $N=1$. 

Now we shall verify the "closer and closer" condition:
\begin{align*}
    |a_{n+1} - a| =  |1^{n+1}-1| =0 = |1^n-1|=|a_n-a|
\end{align*}
Thus, $|a_{n+1}-0|<|a_n-0| \quad \text{for every } n.$ does not hold.
\end{solution}




\question[]
Let $\{b_n\}$ be a bounded sequence and suppose that $a_n \to 0$.
Prove that
\[
a_n b_n \to 0.
\]

\begin{solution}
We wish to prove that \begin{equation*}
    \forall \varepsilon>0,\exists N\in \N s.t. \forall n\ge N, |a_nb_n-0|\le \varepsilon
\end{equation*}. In order to choose such a $N$ we shall use some of those conditions we already know under this question, the $b_n$ is bounded sequence, namely
\begin{equation*}
    \exists M, s.t. \forall n\quad \text{we have } |b_n|\le M
\end{equation*}
which directly implies: $-M\le b_n\le M$ 
and also $a_n\to 0$, which means: \begin{equation*}
    \forall \varepsilon>0,\exists N_1\in \N s.t. \forall n\ge N_1, |a_n-0|\le \varepsilon
\end{equation*}

We can do some algebra on $|a_nb_n-0|\le \varepsilon$:
\begin{align*}
    |a_nb_n-0|&\le \varepsilon \\
    |a_nb_n|&\le \varepsilon\\
    |a_nb_n|&\le |a_n M|=|Ma_n-0|\le \varepsilon \\
\end{align*}
Thus, in order to prove $  \forall \varepsilon>0,\exists N\in \N s.t. \forall n\ge N, |a_nb_n-0|\le \varepsilon$, we must show that under same condition $|Ma_n-0|\le \varepsilon $. Since $a_n\to a$ by Theorem 2.2.4, we know that $Ma_n\to Ma=M*0=0$. Thus, we shall chose $N=N_1$ and then \begin{equation*}
    \forall \varepsilon>0,\exists N_1\in \N s.t. \forall n\ge N_1, |a_nb_n-0|\le \varepsilon
\end{equation*}
Holds.





\end{solution}


\question[] 

\begin{solution}
\end{solution}

\question[] .

\begin{solution}
\end{solution}


\end{questions}

\end{document}
