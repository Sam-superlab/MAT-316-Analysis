\documentclass[12pt,letterpaper, onecolumn]{exam}
\usepackage{amsmath}
\usepackage{tikz}
\usepackage{tikz-cd}
\usepackage{amsthm}

% Number equations by question (nice for homework)
\numberwithin{equation}{question}

% Theorem-like environments, numbered per question
\newtheorem{theorem}{Theorem}[question]
\newtheorem{lemma}[theorem]{Lemma}
\theoremstyle{definition}
\newtheorem{definition}[theorem]{Definition}
\theoremstyle{remark}
\newtheorem{remark}{Remark}

% Make the QED symbol a solid square (optional)
\renewcommand{\qedsymbol}{$\blacksquare$}
\newcommand{\Q}[0]{\mathbb{Q}}
\newcommand{\N}[0]{\mathbb{N}}
\newcommand{\Z}[0]{\mathbb{Z}}

\usepackage{amssymb}
\usepackage[lmargin=71pt, tmargin=1.2in]{geometry}  %For centering solution box
\lhead{ }
\rhead{ }
% \chead{\hline} % Un-comment to draw line below header
\thispagestyle{empty}   %For removing header/footer from page 1

\begin{document}

\begingroup  
    \centering
    \LARGE Anslysis\\
    \LARGE Problem Set 1\\[0.5em]
    \large \today\\[0.5em]
    \large Xuyi (Sam) Ren\par
    % \large Roll Number\par
    \large MAT-316\par
\endgroup
\rule{\textwidth}{0.4pt}
\pointsdroppedatright   %Self-explanatory
\printanswers
\renewcommand{\solutiontitle}{\noindent\textbf{Ans:}\enspace}   %Replace "Ans:" with starting keyword in solution box
\begin{questions}

\question[] 
1.3 question 5 Prove that the set of two-by-two matrices with rational entries is countable
\begin{solution}
Consider the set
\begin{equation}
 M_{2}(\mathbb{Q})
:=
\{
\begin{pmatrix}
a & b \\
c & d
\end{pmatrix}
|
a,b,c,d \in \mathbb{Q}
\}.
\end{equation}
where we wish to prove this set is countable.

By theorem 1.3.5, we know that set of rational numbers is countable. There exists a bijection between $\N$ and $\Q$. 

Now consider the set $\Q^4=\{(a,b,c,d)|a,b,c,d\in \Q\}$, we claim this set is countable.

\begin{proof}
By theorem 1.3.5 we know $\Q$ is countable, by Proposition 1.3.4 we know $\Q\times \Q$ is countable. 

Now consider the set $\Q^4=(\Q\times\Q)\times(\Q\times\Q)$ since $\Q\times \Q$ is countable, again, by proposition 1.3.4 $\Q^4$ is countable.  


\end{proof}

In this case, we must then show that there exists a bijection between $\Q^4$ and $M_2(\Q)$.

Consider the function $f:M_2(\Q)\to \Q^4$ where defined as:

\begin{equation*}
    f(\begin{pmatrix}
a & b \\
c & d
\end{pmatrix})=(a,b,c,d)
\end{equation*}

We claim that this function is a bijection.

\begin{proof}
We prove that $f$ is injective and surjective.

\textbf{Injective}
Let $a,b,c,d,a',b',c',d'$ be arbitrary rational number such that
\[
f\!\left(
\begin{pmatrix}
a & b \\
c & d
\end{pmatrix}
\right)
=
f\!\left(
\begin{pmatrix}
a' & b' \\
c' & d'
\end{pmatrix}
\right).
\]
Then by the definition of $f$,
\[
(a,b,c,d) = (a',b',c',d').
\]
Equality of ordered tuples implies
\[
a=a', \quad b=b', \quad c=c', \quad d=d'.
\]
Therefore,
\[
\begin{pmatrix}
a & b \\
c & d
\end{pmatrix}
=
\begin{pmatrix}
a' & b' \\
c' & d'
\end{pmatrix},
\]
and hence $f$ is injective.

\textbf{Surjective}
Let $(a,b,c,d) \in \mathbb{Q}^4$ be arbitrary.  
Consider the matrix
\[
\begin{pmatrix}
a & b \\
c & d
\end{pmatrix}
\in M_2(\mathbb{Q}).
\]
By the definition of $f$,
\[
f\!\left(
\begin{pmatrix}
a & b \\
c & d
\end{pmatrix}
\right)
=
(a,b,c,d).
\]
Thus every element of $\mathbb{Q}^4$ has a preimage under $f$, and $f$ is surjective.

\medskip

Since $f$ is both injective and surjective, $f$ is a bijection.
\end{proof}

Since $\Q^4$ is countable, we have an bijection $h:\N \to \Q^4$, and a bijection $f:\Q^4 \to M_2(\Q)$. By MAT218 we know that the composition of two bijection is still a bijection, thus, consider the function $f\circ h:\N \to M_2(\Q)$ which is bijection from set of natural numbers to the set of two-by-two matrices.

Therefore, set of two-by-two matrices is countable.
\end{solution}


\question[] 1.3 question 9
For each of of the following sets, say whether it is finite, countable, or uncuntable
\begin{itemize}
    \item The set of functions from a finite set to a finite set.
    \item The set of of functions  from a  finite set to a countable set. 
    \item The  set of  functions from a countable set to finite set with two or more elements.
    \item The set of all finite subsets of the integers. Hint: prove that for each $n$ the set of finite subsets of size $n$ is countable.
\end{itemize}
\begin{solution}
We claim that the set of functions from a finite set to a finite set is finite and countable.

\begin{proof}
Let $|S|=n$ and $|T|=m$. Denote $S$ as
\[
S=\{s_1,s_2,\dots,s_n\}.
\]
since $S$ is finite, it has same cardinality as $\{1,2,3,...,n\}$ for some natural number $n$
A function $f:S\to T$ is completely determined by the choices of the images
\[
f(s_1), f(s_2), \dots, f(s_n),
\]
because once we know $f(s_i)$ for every $i$, we know $f(s)$ for every $s\in S$.

Now count these choices. For each fixed $i\in\{1,2,\dots,n\}$, the value $f(s_i)$ can be any element of $T$, so there are $m$ choices for $f(s_i)$. These choices are independent for different $i$, so by the multiplication principle from MAT218, the total number of possible functions $f:S\to T$ is
\[
\underbrace{m\cdot m\cdots m}_{n\ \text{times}} = m^n.
\]
Therefore $|F|=m^n$, thus $F$ is finite. And since $F$ is finite, it is countable.

\end{proof}

We claim that the set of functions from a finite set to a countable set is infinite and countable. 
\begin{proof}
Denote the set of functions from a finite set to a countable set as $F = \{f \mid f: S \to T\}$, where $S$ is a finite set with $|S| = n > 0$. We denote $S = \{s_1, s_2, \dots, s_n\}$. Since $T$ is a countably infinite set, there exists a bijection $g: \mathbb{N} \to T$.

First, we prove by contradiction that $F$ is infinite. Assume for contradiction that $F$ is finite. Then there exists some natural number $m$ such that $F = \{f_1, f_2, \dots, f_m\}$. Consider the set of values $V = \{f_1(s_1), f_2(s_1), \dots, f_m(s_1)\}$ as a subset of $T^n$. Since $V$ is a finite set (containing at most $m$ elements) and $T$ is infinite, there must exist some element $t^* \in T$ such that $t^* \notin V$. Now, define a function $\phi: S \to T$ as follows:
\begin{equation*}
\phi(s_i) = 
\begin{cases} 
t^* & \text{if } i=1 \\
g(1) & \text{if } i > 1
\end{cases}
\end{equation*}
By construction, $\phi(s_1) = t^*$, but for every function $f_k$ in our list, $f_k(s_1) \in V$. Since $t^* \notin V$, it follows that $\phi(s_1) \neq f_k(s_1)$ for all $k \in \{1, \dots, m\}$. Thus, $\phi \neq f_k$ for any $k$, which contradicts the assumption that $F = \{f_1, \dots, f_m\}$. Therefore, $F$ must be infinite.

Next, we show $F$ is countable by showing $F \subseteq T^n$ and apply proposition 1.3.2 the infinite subset of countable set is countable. Define the mapping $h: F \to T^n$ by:
\begin{equation}
    h(f) = (f(s_1), f(s_2), \dots, f(s_n))
\end{equation}
We prove $h$ is a bijection. 

\textbf{Injective} Suppose $h(f) = h(p)$. Then $(f(s_1), \dots, f(s_n)) = (p(s_1), \dots, p(s_n))$. By the definition of $n$-tuples, $f(s_i) = p(s_i)$ for all $i \in \{1, \dots, n\}$. Since the functions output is the same on all elements of the domain $S$, $f = p$.



\textbf{Surjective} Let $(t_1, t_2, \dots, t_n) \in T^n$ be arbitrary. We can define a function $f: S \to T$ such that $f(s_i) = t_i$ for each $i$. Then $h(f) = (t_1, \dots, t_n)$, so $h$ is surjective.


Next, we show $F$ is countable. We construct an injection from $F$ to the natural numbers $\mathbb{N}$ to treat $F$ as a subset of a countable set. Let $p_1, p_2, \dots, p_n$ be the first $n$ prime numbers (e.g., $2, 3, 5, \dots$). Define $L: F \to \mathbb{N}$ by:
\begin{equation}
   L(f) = p_1^{g(f(s_1))} \cdot p_2^{g(f(s_2))} \cdots p_n^{g(f(s_n))}
\end{equation}
By the Fundamental Theorem of Arithmetic, the prime factorization of any integer is unique. Thus, if $L(f) = L(h)$, the powers of the corresponding primes must be identical s.t. 
\[ g(f(s_i)) = g(h(s_i)) \quad \text{for all } i. \]
Since $g$ is a bijection, this implies $f(s_i) = h(s_i)$ for all $i$, so $f = h$. Therefore, $L$ is injective.

The image of this map, $\text{Im}(L)$, is a subset of $\mathbb{N}$. Since we have established that $F$ is infinite, $\text{Im}(L)$ is an infinite subset of $\mathbb{N}$. By Proposition 1.3.2, any infinite subset of a countable set is countable. Therefore, $\text{Im}(L)$ is countable, which implies $F$ is countable.
\end{proof}

We claim that the set of functions from a countable set to finite set with two or more elements is infinite and uncountable

\begin{proof}
    Again, we shall denote the set of functions from a countable set to finite set with two or more elements as: 
    \begin{equation*}
        F=\{f|f:S\to T\}=\{(s_i,f(s_i))|s_i \in S,i\in \N\}
    \end{equation*}
since $T$ is finite, then there exists some natural number $m\geq 2$ s.t. there exists a bijection between $T$ and $\{1,2,3,...,m\}$, we shall denote $T$ as $T=\{t_1,t_2,t_3,...,t_m\}$, and since $S$ is countable, there exists bijection between $S$ and $\N$. 

Assume for contradiction that $F$ is finite, then there exists some natural number $n$ such that $F=\{f_1,f_2,...,f_n\}$. Now, consider a function $\phi: S \to T$ defined as follows:

\begin{equation*}
\phi(x) = \begin{cases}
\text{a value } t \in T \setminus {f_1(s_1)} & \text{if } x = s_1 \\

\text{a value } t \in T \setminus {f_2(s_2)} & \text{if } x = s_2 \\

\vdots & \vdots \\

\text{a value } t \in T \setminus {f_n(s_n)} & \text{if } x = s_n \\

t_1 & \text{if } x \in {s_{n+1}, s_{n+2}, \dots}

\end{cases}
\end{equation*}

(Note: Since $|T| \ge 2$, the set $T \setminus \{y\}$ is never empty for any $y \in T$, so such a value always exists.)

Such a function is clearly not in $F$ but indeed a function that maps from $S$ to $T$. Thus, a contradiction, $F$ is infinite.

Now we wish to prove $F$ is uncountable. 

That is, there does not exists a bijection between $F$ and $\N$. Assume for contradiction that there exists bijection between $F$ and $\N$. Thus, we have $F=\{f_1,f_2,f_3,...\}$ 

Now, we want to have a construction such that it is a function from $S$ to $T$, but it is not the same of every element in the $F$.

Let's consider such a function $\phi$, since we want it to be different than any function in $F$, meanwhile, such a function must be well defined, that is:
\begin{itemize}
    \item For all $s_i\in S$ there exists an output $t\in T$
    \item  For all $s_i\in S$, the corresponding $t\in T$ is unique.
\end{itemize}
Based on all these necessary features we can construct such a function that swaps the value, so it always have output in $T$, meanwhile, since we didn't "add" a value to the image, the output will still be unique.

\begin{equation*}
    \phi(s_i)=\begin{cases}
    t_2\ \text{If $f_ i(s_i)=t_1$}
\\
t_1 \ \text{If $f_i(s_i)\neq t_1$}
    \end{cases}
\end{equation*}
where $i\in \N$. 

This function is indeed mapping from $S\to T$ where its defined on every $s_i\in S$ so it has an output for each $s_i$. Meanwhile, the output $t_2$ is unique, since by our definition, if there are two outputs, say $t_1,t_2$ for the same $\phi(s_i)$ that would lead to a contradiction $f_i(s_i)=t_1$ and $f_i(s_i)\neq t_1$ both be true.

Meanwhile, this function is not the same as any functions in $F=\{f_1,f_2,f_3,...\}$ since naturally we have $\phi(s_i)\neq f_i(s_i)$ by construction.

Thus, we find a function that is not in the $F$ there is a contradiction. Therefore, $F$ is not countable.
    
\end{proof}

We claim that the set of all finite subsets of the integers is infinite and countable
Hint: prove that for each $n$ the set of finite subsets of size $n$ is countable.

\begin{proof}
    Consider the collection of all finite subsets of integers, denoted as $P$:
    \begin{equation*}
        P:=\{S \mid S \subset \mathbb{Z}, |S| < \infty\}
    \end{equation*}
    

We now show that $P$ is infinite
    
    Consider the sub-collection of sets containing exactly one natural number
    \[ P_\N = \{ \{n\} \mid n \in \mathbb{N} \} \]
    Clearly, $P_\N \subseteq P$ because singleton sets are finite. 
    There exists a bijection $f: P_\N \to \mathbb{N}$ defined by $f(\{n\}) = n$. 
    Since $\mathbb{N}$ is infinite, $P_\N$ is infinite. Since $P$ contains an infinite subset, $P$ is infinite.


We now wish to show that $P$ is countable. As hinted, we first show that for each $n$ the set of finite subsets of size $n$ is countable.

Let
\[
P_i=\{S\mid S\subset \mathbb{Z},\ |S|=i\}.
\]

We prove by induction on $i$.

\begin{itemize}
    \item Base case ($i=1$).  
    $P_1$ is the set of all subsets of $\mathbb{Z}$ with exactly one element.  
    Define a map $f:P_1\to\mathbb{Z}$ by
    \[
    f(\{x\})=x.
    \]
    This map is injective since $f(\{x\})=f(\{y\})$ implies $x=y$, and surjective since for any $z\in\mathbb{Z}$ we have $f(\{z\})=z$. Hence $f$ is a bijection.

    Since $\mathbb{Z}$ is countable, it follows that $P_1$ is countable.

    \item Inductive hypothesis.  
    Suppose $P_k$ is countable. We wish to prove that $P_{k+1}$ is also countable.

    Since $P_k$ is countable, we may enumerate its elements as
    \[
    P_k=\{S_{k_1},S_{k_2},S_{k_3},\dots\},
    \]
    that is, there exists a bijection between $\mathbb{N}$ and $P_k$ sending $i\in\mathbb{N}$ to $S_{k_i}$.

    We now construct $(k+1)$-element subsets by adjoining one additional element to each $S_{k_i}$. Picking such an element directly is tricky, so instead we define the construction uniformly.

    Fix once and for all a bijection $e:\mathbb{N}\to\mathbb{Z}$.  
    For each $S_{k_i}\in P_k$ and each $n\in\mathbb{N}$, consider the set
    \[
    S_{k_i}\cup\{e(n)\},
    \]
    whenever $e(n)\notin S_{k_i}$. This produces a $(k+1)$-element subset of $\mathbb{Z}$.

    In this way, we define a map
    \[
    f:P_k\times\mathbb{N}\to P_{k+1},\qquad
    f(S,n)=S\cup\{e(n)\}.
    \]

Since $P_k$ is countable and $\mathbb{N}$ is countable, the Cartesian product
\[
P_k \times \mathbb{N}
\]
is countable. Consider the subset
\[
A=\{(S,n)\in P_k\times\mathbb{N}\mid e(n)\notin S\}.
\]
Since $A\subseteq P_k\times\mathbb{N}$ and $P_k\times\mathbb{N}$ is countable, it follows that $A$ is countable.

Now define
\[
P_{k+1}=\{\,S\cup\{e(n)\}\mid (S,n)\in A\,\}.
\]
We claim that this construction gives exactly all $(k+1)$-element subsets of $\mathbb{Z}$. Indeed, given any $(k+1)$-element subset $T\subset\mathbb{Z}$, choose an element $x\in T$ and set $S=T\setminus\{x\}$. Then $S\in P_k$, and since $e:\mathbb{N}\to\mathbb{Z}$ is surjective, there exists $n\in\mathbb{N}$ such that $e(n)=x$. Hence $(S,n)\in A$ and $T=S\cup\{e(n)\}$.

Thus $P_{k+1}$ is obtained from a countable set $A$, and therefore $P_{k+1}$ is countable. 
\end{itemize}

By Theorem \ref{theom1} in appendix section we have shown that the countable union of countable sets is countable, thus, $P$ is countable.

\end{proof}

\end{solution}



\question[] section 1.4 question 5
Suppose that $x$ and $y$ satisfy $\frac{x}{2} + \frac{y}{3} = 1$. Prove that $x^2 + y^2 > 1$. \emph{Hint:} Try a contrapositive proof.
\begin{solution}
The contrapositive of the statement is:  $x$ and $y$ satisfy $\frac{x}{2} + \frac{y}{3} \neq 1$. if  $x^2 + y^2 \leq 1$.

\begin{proof}
Assume for contradiction the equation $y=3-\frac{3}{2}x$ holds. Then by subsitution we have
\begin{align*}
    x^2+(3-\frac{3}{2}x)^2 &\leq 1 \\
    \frac{13}{4}x^{2}+9-9x &\leq 1 \\
    \frac{13}{4}x^{2}+8-9x &\leq 0
\end{align*}
Consider the determinant $\Delta=b^2-4ac=9^2-13\times8=-23$, which is negative showing that this qudratic equation $ \frac{13}{4}x^{2}+8-9x =0$ has no real solutions. However,  $x^2 + y^2 \leq 1$ is our premise, since order is not defined on complex numbers, there is a contradiction.

Thus $y\neq 3- \frac{3}{2}x$ that is $\frac{x}{2} + \frac{y}{3} \neq 1$.

    
\end{proof}


\end{solution}


\question[] section 1.4 question 6
Prove that if $n$ is a positive integer, then $n^3 + 5n$ is divisible by $6$.

\begin{solution}
We shall prove by induction. 
\begin{proof}
    \begin{itemize}
\item Base case(n=1): substitute $n=1$, then we will have $n^3 + 5n=6$ which is divisible by 6 
\item Inductive hypothesis: Suppose that $n=k$ holds for the statement, where $k^3+5k$ is divisible by 6, namely $k^3+5k=6a$ for some natural number $a$, we wish to prove that for $k+1$, we have $(k+1)^3+5(k+1)$ also divisible by 6.
Consider the expression $  (k+1)^3+5(k+1) = k^3+5k+(3k^2+3k+6)$ where we already now that $k^3+5k$ is divisible by 6, thus, we wish to show that $(3k^2+3k+6)$ is also divisible by 6. We have two cases for $k$, it's either odd or even. If $k$ is odd, then we must have $3k^2$  and $3k$ both be odd, and the sum of two odd number would be even number. Namely, $3k^2+3k$ is divisible by two, meanwhile, notice that it is also divisible by 3 since $3k^2+3k=3(k^2+k)$, thus, $3k^2+3k+6$ is divisible by 6. Therefore, when $k$ is odd, our statement holds. Consider the case when $k$ is even, then we must have $k^2$ and $k$ both be even, meanwhile, the whole expression($3k^2+3k$) would be even, in a similar reason, we have $3k^2+3k+6$ be dividibe by both 2,3, that is, 6. Therefore, we have showed that $ (k+1)^3+5(k+1)$ is divisible by 6 in each cases. Thus, induction step holds.


\end{itemize}

As a conclusion, if $n$ is a positive integer, then $n^3 + 5n$ is divisible by $6$.
\end{proof}

\end{solution}

\question[] section 1.4 question 8
Prove that for all positive integers $n$, 
$1^3 + 2^3 + \cdots + n^3 = (1 + 2 + 3 + \cdots + n)^2$.

\begin{solution}
We wish to prove 
\begin{equation*}
    \sum_{i=1}^n i^3 = (\sum_{i=1}^n i)^2
\end{equation*}
We prove by induction:

\begin{itemize}
    \item Base case($n=1$): LHS=$1^3=1$, RHS=$1^2=1$, thus $\sum_{i=1}^1 i^3 = (\sum_{i=1}^1 i)^2$
    \item Induction hypothesis: Suppose $\sum_{i=1}^k i^3 = (\sum_{i=1}^k i)^2$ holds. We wish to prove $\sum_{i=1}^{k+1} i^3 = (\sum_{i=1}^{k+1} i)^2$.

Consider LHS, we shall expand it:
\begin{align*}
    \sum_{i=1}^{k+1} i^3 &= \sum_{i=1}^{k}i^3+(k+1)^3\\
    &= \sum_{i=1}^{k} i^3 + k^3+3k^2+3k+1
\end{align*}
Now let's consider the RHS, again we expand and modify it with some algebra:
\begin{align*}
    (\sum_{i=1}^{k+1}i)^2&=(\sum_{i=1}^{k}i+k+1)^2 \\
    &=(\sum_{i=1}^{k}i)^2 + 2k \sum_{i=1}^{k}i + 2\sum_{i=1}^{k}i + k^2+2k+1    \ (\text{Expand the expression})     \\
    &=(\sum_{i=1}^{k}i)^2 + 2(k+1) \sum_{i=1}^{k}i  + k^2+2k+1 \\
    &= (\sum_{i=1}^{k}i)^2 + 2(k+1) \frac{k(k+1)}{2}  + k^2+2k+1  \\ (&\text{Apply the equation from MAT218 that $\sum_{i=1}^ni=\frac{n(n+1)}{2}$})\\
    &= (\sum_{i=1}^{k}i)^2 + k(k+1)^2  + k^2+2k+1 \\
    &= (\sum_{i=1}^{k}i)^2+ k^3  + 3k^2+3k+1 \\
    &= \sum_{i=1}^{k} i^3+ k^3  + 3k^2+3k+1=LHS  \ (\text{By induction hypothesis})
\end{align*}
Thus, $1^3 + 2^3 + \cdots + n^3 = (1 + 2 + 3 + \cdots + n)^2$ holds for all $n$
\end{itemize}
\end{solution}

\section{Appendix}

\begin{theorem}\label{theom1} A countable union of countable sets is countable
    
\end{theorem}
\begin{proof}
    We now prove that a countable union of countable sets is countable.

Let $\{A_n\}_{n\in\mathbb{N}}$ be a collection of countable sets.  
Since each $A_n$ is countable, for every $n\in\mathbb{N}$ there exists a surjection
\[
f_n:\mathbb{N}\to A_n.
\]

Consider the Cartesian product $\mathbb{N}\times\mathbb{N}$, which is countable by Proposition 1.3.4.
Define the set
\[
B=\{(n,m)\in\mathbb{N}\times\mathbb{N}\mid f_n(m)\in A_n\}.
\]
Clearly, $B=\mathbb{N}\times\mathbb{N}$, and hence $B$ is countable.

Now define a mapping
\[
\phi:B\to \bigcup_{n=1}^{\infty} A_n
\]
by
\[
\phi(n,m)=f_n(m).
\]

We claim that
\[
\bigcup_{n=1}^{\infty} A_n \subseteq \phi(B).
\]
Indeed, let $x\in \bigcup_{n=1}^{\infty} A_n$. Then there exists some $k\in\mathbb{N}$ such that $x\in A_k$.
Since $f_k$ is surjective, there exists $m\in\mathbb{N}$ such that $f_k(m)=x$.
Thus $(k,m)\in B$ and $\phi(k,m)=x$.

Therefore, $\bigcup_{n=1}^{\infty} A_n$ is a subset of $\phi(B)$.
Since $\phi(B)\subseteq \bigcup_{n=1}^{\infty} A_n$, we conclude that
\[
\bigcup_{n=1}^{\infty} A_n = \phi(B).
\]

Since $B$ is countable and $\bigcup_{n=1}^{\infty} A_n$ is an infinite subset of a countable set, it follows that
\[
\bigcup_{n=1}^{\infty} A_n
\]
is countable.

\end{proof}


\end{questions}

\end{document}
