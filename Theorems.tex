\documentclass[10pt]{article}

% ---------- Page Layout ----------
\usepackage[
  paper=a4paper,
  left=0.8in,
  right=2.2in, % space for notes
  top=0.8in,
  bottom=0.8in
]{geometry}

% ---------- Math Packages ----------
\usepackage{amsmath, amssymb, amsthm}

% ---------- Marginal Notes ----------
\usepackage{marginnote}
\renewcommand*{\marginfont}{\footnotesize}
\setlength{\marginparwidth}{1.7in}

% ---------- Formatting ----------
\setlength{\parindent}{0pt}
\setlength{\parskip}{6pt}

% ---------- Theorem Styles ----------
\newtheoremstyle{examthm}
  {6pt}   % space above
  {6pt}   % space below
  {}      % body font
  {}      % indent
  {\bfseries} % head font
  {.}     % punctuation
  {0.5em} % space after head
  {}

\theoremstyle{examthm}
\newtheorem{theorem}{Theorem}
\newtheorem{lemma}{Lemma}
\newtheorem{proposition}{Proposition}
\newtheorem{corollary}{Corollary}

% ---------- Custom Note Command ----------
\renewcommand{\thmnote}[1]{\marginnote{\emph{#1}}}
\newcommand{\proofnote}[1]{\marginpar{\normalsize\emph{#1}}}

\newcommand{\sep}{%
  \vspace{0.6em}
  \noindent\rule{\linewidth}{0.4pt}
  \vspace{0.6em}
}


\newcommand{\R}{\mathbb{R}}
\newcommand{\N}{\mathbb{N}}
\newcommand{\Z}{\mathbb{Z}}
\newcommand{\Q}{\mathbb{Q}}
% ---------- Document ----------
\begin{document}

\begin{center}
    {\Large \textbf{Theorem List}}\\
    \vspace{0.3em}
    {\small Analysis \quad • \quad Spring 2026 \quad • \quad Xuyi Ren}
\end{center}

\vspace{0.8em}

% ==================================================

\section*{Countability}

\begin{proposition}[1.3.1]
    Let $S$, $T$, and $U$ be sets.  
If $S$ and $T$ have the same cardinality and $T$ and $U$ have the same cardinality,
then $S$ and $U$ have the same cardinality.
\proofnote{Use composition of bijections.}
\end{proposition}
\begin{proof}
    Since $S$ and $T$ have the same cardinality, there exists a one-to-one function
$f : S \to T$.  
Since $T$ and $U$ have the same cardinality, there exists a one-to-one function
$g : T \to U$.

Then $g \circ f : S \to U$ is one-to-one.  
Therefore, by definition, $S$ and $U$ have the same cardinality.
\proofnote{Injective $\circ$ injective = injective}
\end{proof}






\sep


\begin{proposition}[1.3.2]
    If $S$ is infinite subset of a countable set $T$, then $S$ is countable
\end{proposition}
\begin{proof}
    Since $T$ is countable, there exists a bijection between $\N$ and $T$. The element of $S$ is a subset of $T=\{f(n)|n\in \N\}$. 

    \proofnote{Similar how we define the subsequence, we need a function map $k$ to $n_k$}

    Let $n_1$ be the smallest integer in $\N$ such that $f(n_1)\in S$. Similarly, we define $n_k$ be the smallest integer bigger than $n_{k-1}$ such that $f(n_k)\in S$. It is easy to check that the function $g$ defined as
    \begin{equation*}
        k   \xrightarrow{g} f(n_k)
    \end{equation*}
    is a bijection between $\N$ and $S$. Therefore, $S$ is countable.
\end{proof}



\sep




\begin{proposition}[1.3.4]
    The cartesian product of countable set is countable.
\end{proposition}
\begin{proof}
    Let $S$ and $T$ be countable arbitrary set. Since they are countable, we have two bijections $f,g$ from $\N$ to $S$ and $T$ respectivitly. 

    Each element of $S\times T$ has form $(f(n),g(m))$ where $n,m\in \N$. Let $h:S\times T \to \N$  defined as:

    \begin{equation*}
        h((f(n),g(m)))=2^n3^m
    \end{equation*}
    Then by fundamental theorem of arithmetic, $h$ is a bijection between $S\times T$ and a infinite subset of $\N$. By proposition 1.3.2 such a subset is countable, by proposition 1.3.1, $S\times T$ is countable.
\end{proof}








\sep
\begin{theorem}[1.3.5]
    The set of rational number is countable
\end{theorem}
\begin{proof}
    Every rational number can be written in the form $\frac{m}{n}$ where $m,n$ are integers and $n\neq 0$ and $\gcd(m,n)=1$. The function $f$ defined by 
    \begin{equation*}
        \frac{m}{n} \xrightarrow{f} (m,n)
    \end{equation*}
    is a bijection between positive rational number and $\N\times \N$. By proposition 1.3.4, we know that $\N\times\N$ is countable, by proposition 1.3.1, we know the set of positive rational number is countable. 
    Since $\Q$ can be express as $\Q_-\cup\{0\}\cup\Q_+$, thus we shall define a bijection $g:\Q_+\to \Q_-\cup\{0\}$ s.t.
    \begin{equation*}
        g(q)=\begin{cases}
            0 \quad \text{if}\ q=0\\
            -q  \quad \text{if}\ q\in \Q_+
        \end{cases}
    \end{equation*}
    thus, $\Q_-\cup\{0\}$ is countable, and thus, $\Q$ is countable.
\end{proof}



\sep
\begin{theorem}[1.3.6]
    The set of all real numbers between 0 and 1 is uncountable.
\end{theorem}

\begin{proof}
We proceed by contradiction. Suppose that the open interval $(0,1)$ is countable.  
Then there exists a one-to-one function
\[
f : \mathbb{N} \to (0,1)
\]
whose range is $(0,1)$.

Every real number in $(0,1)$ admits a decimal expansion
\[
x = 0.x_1 x_2 x_3 \cdots
\]
where each digit $x_i \in \{0,1,\dots,9\}$.  
We exclude decimal expansions that terminate in infinitely many $0$'s or $9$'s so that each real number has a unique representation.
\proofnote{This avoids ambiguities such as $0.5000\ldots = 0.4999\ldots$}

Write the decimal expansion of each $f(n)$ as
\[
\begin{aligned}
f(1) &= 0.x^{(1)}_1 x^{(1)}_2 x^{(1)}_3 \cdots \\
f(2) &= 0.x^{(2)}_1 x^{(2)}_2 x^{(2)}_3 \cdots \\
f(3) &= 0.x^{(3)}_1 x^{(3)}_2 x^{(3)}_3 \cdots \\
&\ \ \vdots \\
f(n) &= 0.x^{(n)}_1 x^{(n)}_2 x^{(n)}_3 \cdots
\end{aligned}
\]

We now construct a new decimal number
\[
y = 0.y_1 y_2 y_3 \cdots
\]
by choosing each digit $y_n \in \{2,3,\dots,8\}$ such that
\[
y_n \neq x^{(n)}_n .
\]
\proofnote{Choosing digits between $2$ and $8$ ensures the decimal does not end in $0$'s or $9$'s}

By construction, $y \in (0,1)$. Moreover, $y$ differs from $f(n)$ in the $n$th decimal place for every $n$, hence
\[
y \neq f(n) \quad \text{for all } n \in \mathbb{N}.
\]

Thus $y$ is a real number in $(0,1)$ that is not in the range of $f$, contradicting the assumption that $f$ is surjective onto $(0,1)$.

Therefore, $(0,1)$ is uncountable.
\proofnote{In contrast, $\mathbb{Q} \cap (0,1)$ \emph{is} countable; the diagonal argument crucially uses infinite decimal expansions}
\end{proof}





\section*{Sequence}
===============================================

\section*{Integral}
============================================

\end{document}
